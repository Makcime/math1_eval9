\ifeabs{\question[4]}{\question[6]}
On a construit un carré $ABCD$ de côté 4 cm. \\ \textbf{COMPLÈTE} la figure en traçant :
\begin{enumerate}
	\item Un cercle de centre $\versions{A}{B}{C}{D}$ qui passe par le point $\versions{B}{A}{C}{D}$.
	\item Un cercle de centre $\versions{A}{B}{C}{D}$ et de rayon $\versions{5}{6}{7}{5}$ cm.
\end{enumerate}

\vspace{4cm}

\begin{center}
	\begin{tikzpicture}[scale=1]
		% Carré ABCD de côté 4 (unités)
		\coordinate (A) at (0,0);
		\coordinate (B) at (4,0);
		\coordinate (C) at (4,4);
		\coordinate (D) at (0,4);

		\draw[thick] (A) -- (B) -- (C) -- (D) -- cycle;

		\node[below left]  at (A) {$A$};
		\node[below right] at (B) {$B$};
		\node[above right] at (C) {$C$};
		\node[above left]  at (D) {$D$};

		% Indication de la longueur AB
		\draw[<->] (A) ++(0,-0.6) -- node[below]{4 cm} ($(B)+(0,-0.6)$);

		% Cercles en rouge pour le correctif uniquement
		\ifcorr{
			{\color{red}
					\versions{
						% Version A : centre A, cercle passant par B (rayon 4) et rayon 5
						\draw[thick,red] (A) circle (4);
						\draw[thick,red] (A) circle (5);
					}{
						% Version B : centre B
						\draw[thick,red] (B) circle (4);
						\draw[thick,red] (B) circle (6);
					}{
						% Version C : centre C
						\draw[thick,red] (C) circle (4);
						\draw[thick,red] (C) circle (7);
					}{
						% Version D : centre D
						\draw[thick,red] (D) circle (4);
						\draw[thick,red] (D) circle (5);
					}
				}
		}
	\end{tikzpicture}
\end{center}
